%%%%%请在下一行设定文档类为article%%%%%
\documentclass{article}
%%%%%%%%%%%%%%%%%%%%%%%%%%%%%%%%%%%

\usepackage{fontspec, xunicode, xltxtra}	%提供英文字体设置的宏包
\setmainfont{Times New Roman}				%设置英文主字体 Times New Roman

\usepackage{xeCJK}				%提供中文支持的宏包
\setCJKmainfont{SimSun}			%设定主字体为宋体

\usepackage{indentfirst}		%设定首段也缩进,符合中文习惯
\setlength{\parindent}{2em}		%设定段首行空两个字符宽

\linespread{1.2} 				%设置行距

\renewcommand{\figurename}{图}	%将 `figure' 转为 ``图''

%%%%%请在接下来几行设定文章标题和作者%%%%%
%标题:OOP-QUIZ-2参考程序说明文档}
\title{OOP-QUIZ-2参考程序说明文档}

%作者:自己的名字
\author{朱祺}

%%%%%%%%%%%%%%%%%%%%%%%%%%%%%%%%%%%%%

\begin{document}
%%%%%请在下一行生成标题%%%%%
%%%%%%%%%%%%%%%%%%%%%%%%%
\maketitle
%%%插入一个名为“问题描述”的section
\section{问题描述}
通过对比《程序设计基础》课程中“下楼问题”和“跳马问题”的源程序,
我们能发现它们属于同一类问题,具有一定的共性。

现需要运用所学的OOP思想,重新设计和实现由多个类组成的类体系架构,能够处理该类问题的求解。

在源代码中,需要包含对类的测试代码,请以下楼问题和跳马问题的求解来进行测试。

%%%插入一个名为“问题分析”的section
\section{问题分析}

通过分析这两类问题,发现他们具有如下共同点:

%%%%%请开始一个1. 2. 3. ... 的环境%%%%%
\begin{enumerate}
	\item 需要寻找起始状态到最终状态的路径。
	\item 可能的变化形式只有有限种,且不会导致重复。
	\item 采用深度优先搜索算法。
	\item 需要输出所有的可能路径。
	
	% 1. 需要寻找起始状态到最终状态的路径。
	% 2. 可能的变化形式只有有限种,且不会导致重复。
	% 3. 采用深度优先搜索算法。
	% 4. 需要输出所有的可能路径。

%%%%%请结束这个1. 2. 3. ... 的环境%%%%%
\end {enumerate}
因此可以将这类问题归纳为一个基类,两个具体问题分别是两个派生类,
都由这个基类继承而来。
\section{类的建立}
%%%插入一个名为“类的建立”的section
\subsection{Problem类}
%%%插入一个名为“Problem类”的subsection

Problem类是此类问题的基类。主要包括以下成员函数
\begin{description}
%%%%%请开始一个描述环境%%%%%
	\item[void solve()] 求解问题的函数。
	\item[virtual void move(int dir)] 向dir方向移动,
		对应于搜索中的加深一步。
	\item[virtual void moveBack()] 移回上一步,
		对应于搜索中的回溯。
	\item[virtual bool isLegal()] 判断位置是否合法。
	\item[virtual bool targetReached()] 判断是否到达目标。
	\item[void print()] 输出解法。
%%%%%请结束这个描述环境%%%%%
\end {description}
步骤在程序中用vector<int>型变量表示。
\subsection{KnightProblem类和StairProblem类}
%%%插入一个名为“KnightProblem类和StairProblem类”的subsection

具体实现在此省略。最后形成的类图如图\ref{fig:figure1}所示。

\begin{figure}[htbp]
	\centering
	\includegraphics[]{16222_100.png}
	\caption{类图}
	\label{fig:figure1}
\end{figure}

%%%插入一个名为“程序测试”的section
\section{程序测试}
我这种天才写的程序怎么可能会有错呢?哈哈哈哈哈!

其实这只是一个\XeTeX 的教学而已,大家不要太当真:)

\end{document}

